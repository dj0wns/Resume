\documentclass[10pt,oneside]{article}
\usepackage{geometry}
\usepackage[T1]{fontenc}
\usepackage{amsfonts}
\usepackage{etoolbox}
\usepackage{multicol}
\usepackage{enumitem}

\pagestyle{empty}
\geometry{letterpaper,tmargin=0.6in,bmargin=0.6in,lmargin=.85in,rmargin=.85in,headheight=0in,headsep=0in,footskip=.3in}

\setlength{\parindent}{0in}
\setlength{\parskip}{0in}
\setlength{\itemsep}{0in}
\setlength{\topsep}{0in}
\setlength{\tabcolsep}{0in}

% Name and contact information
\newcommand{\name}{Derek Jones}
\newcommand{\homeaddrtop}{}
\newcommand{\homeaddrbot}{}
\newcommand{\schooladdrtop}{APT 2404}
\newcommand{\schooladdrmid}{145 Ames St}
\newcommand{\schooladdrbot}{Marlborough, MA 01752}
\newcommand{\cellphone}{(925) 348-0232}
\newcommand{\email}{DerekJones@asu.edu}
\newcommand{\github}{http://www.github.com/dj0wns}
%%%%%%%%%%%%%%%%%%%%%%%%%%%%%%%%%%%%%%%%%%%%%%%%%%%%%%%%%
% New commands and environments

% This defines how the name looks
\newcommand{\bigname}[1]{
  \begin{center}\huge\scshape#1\end{center}
}

%\fontfamily{phv}
\newcommand{\sectitle}[1]{
  \begin{flushleft}{\selectfont\Large#1}\end{flushleft}
}

% A ressection is a main section (<H1>Section</H1>)
\newenvironment{ressection}[1]{
  \vspace{2pt}
  \sectitle{#1}
  \vspace{-10pt}\rule{\textwidth}{0.5pt}
  \vspace{-10pt}
  \begin{itemize}[leftmargin=13pt]
  \vspace{3pt}
}{
  \end{itemize}
}

% A resitem is a simple list element in a ressection (first level)
\newcommand{\resitem}[1]{
  \vspace{-4pt}
  %\item \begin{flushleft} #1 \end{flushleft}
  \item \begin{flushleft} #1 \end{flushleft}
  %\item[\(\circ\) ] \begin{flushleft} #1 \end{flushleft}
}

% A ressubitem is a simple list element in anything but a ressection (second level)
\newcommand{\ressubitem}[1]{
  \vspace{-1pt}
  \item \begin{flushleft} #1 \end{flushleft}
}

% A resmeditem is a complex list element for stuff like jobs and education:
%  Arg 1: Name of company or university
%  Arg 2: Location
\newcommand{\resmeditem}[2]{
  \vspace{-5pt}
  \item
  %\item
  \textbf{#1}---#2
}

% A resbigitem is a complex list element for stuff like jobs and education:
%  Arg 1: Name of company or university
%  Arg 2: Location
%  Arg 3: Title and/or date range
\newcommand{\resbigitemline}[3]{
  \vspace{-5pt}
  \item
  %\item
  \textbf{#1}---#2, 
  \textit{#3}
}

% A resbigitem is a complex list element for stuff like jobs and education:
%  Arg 1: Name of company or university
%  Arg 2: Location
%  Arg 3: Title and/or date range
\newcommand{\resbigitem}[3]{
  \vspace{-5pt}
  \item
  %\item
  \textbf{#1}---#2 \\
  \textit{#3}
}

% A resititem is a list element that just does an italic item for the sole
% purpose of expanding jobs at the same company
\newcommand{\resititem}[1]{
  \vspace{-5pt}
  \textit{#1}
}

% This is a list that comes with a resbigitem
\newenvironment{ressubsec}[3]{
  \resbigitem{#1}{#2}{#3}
  \vspace{-4pt}
  \begin{itemize}[leftmargin=*]

}{
  \end{itemize}
}
% This is a list that comes with a resbigitem
\newenvironment{ressubsecshort}[2]{
  \resmeditem{#1}{#2}
  \vspace{-4pt}
  \begin{itemize}
  }{
  \end{itemize}
}

%List with inline resbigitem
\newenvironment{ressubsecline}[3]{
  \resbigitemline{#1}{#2}{#3}
  \vspace{-2pt}
  \begin{itemize}
}{
  \end{itemize}
}

% This is a simple sublist
\newenvironment{reslist}[1]{
  \resitem{\textbf{#1}}
  \vspace{-5pt}
  \begin{itemize}
}{
  \end{itemize}
}

% A resreference is a references section
\newenvironment{resreference}[2]{
  \vspace{14pt}
  {\fontfamily{phv}\selectfont\Large#1}


  \vspace{10pt}


  \begin{tabular}{#2}
  \vspace{0pt}
}{
  \end{tabular}
}

% reduce spacing between lines


\AfterEndEnvironment{ressubsec}{\vspace{-.2\baselineskip}}

\AfterEndEnvironment{ressubsecshort}{\vspace{-.5\baselineskip}}

\AfterEndEnvironment{ressubitem}{\vspace{-.5\baselineskip}}

\AfterEndEnvironment{resitem}{\vspace{-.5\baselineskip}}
\BeforeBeginEnvironment{ressection}{\vspace{.2\baselineskip}}
\AfterEndEnvironment{ressection}{\vspace{-.2\baselineskip}}
%%%%%%%%%%%%%%%%%%%%%%%%%%%%%%%%%%%%%%%%%%%%%%%%%%%%%%%%%
% Now for the actual document:

\begin{document}

\fontfamily{ppl} \selectfont

% Name with horizontal rule
\bigname{\name}
\vspace{-6pt} \rule{\textwidth}{1pt}
\vspace{-10pt}

\begin{center}
% \begin{multicols}{3}
  \cellphone \\
  \email \\
  \github \\
% \end{multicols}
\end{center}

\vspace{-26 pt}




%%%%%%%%%%%%%%%%%%%%%%%%
\vspace{\baselineskip}
\begin{ressection}{Education}
  \resbigitem{Bachelor of Science in Computer Science}{Arizona State University, Tempe, AZ}{May 2017}
  % \ressubitem{Ira A. Fulton Schools of Engineering}
  %\end{ressubsec}
\end{ressection}


%%%%%%%%%%%%%%%%%%%%%%%%
\begin{ressection}{Qualifications}

  \resitem{Experience using Git and bug trackers in a production environment}

  \resitem{Strong written and verbal communication skills developed through team projects and presentations}

  \begin{reslist}{Computer Languages and Environments:}
    \ressubitem{Proficient in C and C++}
    \ressubitem{Experience with Java, OpenMP, MPI, Bash, Gnuplot, and LaTeX}
    \ressubitem{Exposure to Python, Go, Javascript, SQL, and Matlab}
  \end{reslist}

  \resitem{\textbf{Operating Systems:} Windows,
  UNIX/Linux (Arch, Redhat, Ubuntu)}

\end{ressection}

%%%%%%%%%%%%%%%%%%%%%%%%
\begin{ressection}{Work Experience}
  \begin{ressubsec}{Electrical Engineer}{Raytheon, Radar Signal Processing}{June 2017 -- Present}
    \ressubitem{Software development for next generation Radar Signal Processing applications utilizing a distributed high performance computing architecture with strong emphasis on vectorization}
    \ressubitem{Development consists of C++11 in Linux using Totalview, Boost.test, and Rational Clearcase}
  \end{ressubsec}
  
  \begin{ressubsec}{Software Engineering Intern}{Lawrence Livermore National Laboratory, High Energy Density Physics}{May 2016 -- August 2016}
    \ressubitem{Implemented and analyzed various acceleration structures for use within LLNL's Monte Carlo Particle Transport Code, Mercury}
  \end{ressubsec}
  
  \begin{ressubsec}{Software Engineering Intern}{ViaSat Inc.}{May 2015 -- August 2015}
    \ressubitem{Designed, implemented, and tested an Android collaboration application tailored for operation over satellite networks}
  \end{ressubsec}

\end{ressection}



%%%%%%%%%%%%%%%%%%%%%%%%
\begin{ressection}{Projects}
  \begin{ressubsecline}{Virtual Reality Visualization of Monte Carlo Particle Transport}{Honors Thesis}{C++ - 1 Person}
    \ressubitem{In a collaborative effort with the Lawrence Livermore National Laboratory, I created a virtual reality visualization of the particle transport code Mercury, utilizing an HTC Vive}
    \ressubitem{Developed a system for generating three dimensional primitives in Unity, modified a constructive solid geometry library to allow the creation of more complex shapes, and implemented HTC vive support}
  \end{ressubsecline}
  \begin{ressubsecline}{RAID-Like Cloud Storage}{Pennapps XV - Top 30}{C++/Python - 3 People}
    \ressubitem{Created a mountable virtual drive for distributing, encrypting, and retrieving files across multiple cloud storage providers}
    \ressubitem{Developed the FUSE filesystem implementation, added encryption and decryption functionality, developed the simulated RAID method of file splitting, and created the hooks for sending and receiving data}
  \end{ressubsecline}

\end{ressection}



%%%%%%%%%%%%%%%%%%%%%%%%
\begin{ressection}{Competitions}
  \begin{ressubsecline}{SuperComputing 15 Conference}{Student Cluster Competition 2015}{Arizona Tri-University Team}
    \ressubitem{Collaborated with four other students to compile and run LINPACK, Trinity, WRF, MILC, HPC Repast and
    HPCG in a UNIX environment using the Slurm workload manager}
  \ressubitem{Competed to compute the provided data sets in the fastest time over a three day period}
  \end{ressubsecline}
  \begin{ressubsecshort}{ASU Programming Competition 2016}{1\textsuperscript{st} Place Overall}
    \ressubitem{Collaborated with two teammates to solve logic problems in C++}
  \end{ressubsecshort}
\end{ressection}
\end{document}
